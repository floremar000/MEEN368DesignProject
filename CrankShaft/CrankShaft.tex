\documentclass[10pt,a4paper]{article}
\usepackage[utf8]{inputenc}
\usepackage{amsmath}
\usepackage{amsfonts}
\usepackage{amssymb}
\usepackage{gensymb}
\begin{document}
\section*{Crank Shaft Calculations}
\begin{itemize}
	\item $l_c$ is the length of each crank.
	\item $l_s$ is the length of each shaft section between each crank.
	\item $t_b$ is the thickness of shaft crank connectors.
	\item $l_a$ is the length of each crank \& shaft subsection. $l_a = l_s + 2t + l_c$
	\item $l_b$ is the length of the shaft between each bearing location and closest crank connector.
	\item $l_s$ is the length of the shaft between the two bearings.

\end{itemize}
The crankshaft consists of six separate pistons acting on the shaft. Since each piston undergoes 4 strokes, a total cycle of $4 \pi$ is needed. Thus each piston stroke is offset by 120 degrees. Piston $n$ is the $n$th piston from the bearing next to the flywheel. Each force on the crank shaft is a vector equation given below.
\begin{align}
	\vec{F}_n &= f(\theta + 120 \degree (n-1) )
\end{align}
The direction of each crank shaft is offset by 120 degrees too, so the piston cycles are offset by 120 degrees. The radius vector of each crank is given below.
\begin{align}
	\vec{r}_n = r_c \Big \langle \cos (\theta + 120 \degree (n-1)  )\ \hat{i},\ \sin (\theta + 120 \degree (n-1))\ \hat{j}  \Big \rangle
\end{align}
For the shaft coordinates, the $\hat{k}$ vector is parallel to the axis of the shaft, from the first bearing to the second. The torque equation along the shaft is simply written out using cross products.
\begin{align}
	\vec{T}(x) &= \sum_{n=1}^6 \langle x - l_a (n-1) - l_b - \frac{l_c}{2} - t\rangle^0\ \vec{F}_n \times \vec{r}_n 
\end{align}
There are two reaction forces at each of the bearings, $\vec{R}_1$ and $\vec{R}_2$. The force and moment equations are written out to solve for these.
\begin{align}
	\sum \vec{F} &= \vec{R}_1 + \vec{R}_2 + \sum_{n=1}^6 \vec{F}_n = 0 \\
	\sum \vec{M} &= \vec{R}_2 \times l_s \hat{j} + \sum_{n=1}^6 \vec{F}_n \times ( x - l_a (n-1) - l_b - \frac{l_c}{2} - t)\ \hat{j} = 0\\
	\vec{R}_2 &= - \sum_{n=1}^6 \frac{\vec{F}_n ( x - l_a (n-1) - l_b - \frac{l_c}{2} - t)}{l_s}\\
	\vec{R}_1 &= \sum_{n=1}^6 \frac{\vec{F}_n ( x - l_a (n-1) - l_b - \frac{l_c}{2} - t)}{l_s} - \sum_{n=1}^6 \vec{F}_n
\end{align}
The shear and moment equations are derived below. The shear and moment is not affected by the crank offset, since there are no axial forces.
\begin{align}
	\vec{V}(x) &= \vec{R}_1 \langle x \rangle^0 + \sum_{n=1}^6 \vec{F}_n\ \langle x - l_a (n-1) - l_b - \frac{l_c}{2} - t\rangle^0\\
	\vec{M}(x) &= \vec{R}_1 \langle x \rangle + \sum_{n=1}^6 \vec{F}_n\ \langle x - l_a (n-1) - l_b - \frac{l_c}{2} - t\rangle\\
\end{align}
The maximum moment and shear stresses are given. The shear stress equation only applies for the on axis shaft.
\begin{align}
	\sigma_M (x) = \pm \frac{|\vec{M}(x)| d(x)}{2 I(x)}\\
	\tau (x) = \frac{|T(x)| d (x)}{2 J(x)}
\end{align}
\end{document}